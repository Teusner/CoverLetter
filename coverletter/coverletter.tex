%!TEX TS-program = xelatex
%!TEX encoding = UTF-8 Unicode

\documentclass[11pt, a4paper]{awesome-cv}

\usepackage{pgfgantt}
\usepackage{graphicx}
\usepackage{xcolor}
\usepackage{enumitem}

\geometry{left=1.4cm, top=.8cm, right=1.4cm, bottom=1.8cm, footskip=.5cm}
\fontdir[fonts/]
\colorlet{awesome}{awesome-skyblue}
\setbool{acvSectionColorHighlight}{true}

% If you would like to change the social information separator from a pipe (|) to something else
\renewcommand{\acvHeaderSocialSep}{\quad\textbar\quad}

% Available options: circle|rectangle,edge/noedge,left/right
% \photo[circle,noedge,left]{./examples/profile}
\name{Quentin}{Brateau}
\position{Robotics Engineer{\enskip\cdotp\enskip}Embedded Systems}
\address{17 rue Adolphe Laberte 51100 Reims}

\mobile{(+33) (0)6 04 18 24 28)}
\email{quentin.brateau@ensta-bretagne.org}
\homepage{teusner.github.io/portfolio}
\github{Teusner}
\linkedin{quentinbrateau}
% \gitlab{gitlab-id}
% \stackoverflow{SO-id}{SO-name}
% \twitter{@twit}
% \skype{skype-id}
% \reddit{reddit-id}
% \medium{medium-id}
% \googlescholar{googlescholar-id}{name-to-display}
%% \firstname and \lastname will be used
% \googlescholar{googlescholar-id}{}
% \extrainfo{extra informations}

% \quote{``Be the change that you want to see in the world."}

% The company being applied to
\recipient
  {\textsc{Tampere University}}
  {Kalevantie 4, 33100 Tampere}
% The date on the letter, default is the date of compilation
\letterdate{\today}
% The title of the letter
\lettertitle{Ph.D. Letter of Intent : Safe and resilient control of autonomous surface vessels in dynamic environments}
% How the letter is opened
\letteropening{Dear Madam/Sir,}
% How the letter is closed
\letterclosing{I hope you will consider my application favourably and I look forward to hearing from you. \\[\baselineskip] Yours sincerly,}
% Any enclosures with the letter
% \letterenclosure[Pièces Jointes]{
% 	\begin{itemize}
% 		\item Curriculum Vitae
% 		\item Lettre de recommandation du professeur Luc \textsc{Jaulin}
% 		\item Lettre de recommandation d'Alaa \textsc{El Jawad} tuteur de mon projet de fin d'études
% 		\item Relevés de notes obtenus à l'\textsc{ENSTA} Bretagne
% 	\end{itemize}~}

\begin{document}

	% Give optional argument to change alignment(C: center, L: left, R: right)
	\makecvheader[C]

	\makecvfooter{\today}{Quentin Brateau~~~·~~~Letter of Intent}{}

	\makelettertitle

	\begin{cvletter}

		I am Quentin Brateau, a French engineer in robotics. I was graduated in 2021 ENSTA Bretagne, in Brest, a prominent French graduate, post-graduate school, Research Institute, and doctoral college, which trains engineers in various fields of engineering such as Robotics, which is my specialty. I also was graduated the same year with a master of science in dynamical systems and signals from the University of Angers in France.

		My academic training has a strong emphasis on Science because I have completed a two-year intensive post-secondary school preparation for the competitive entrance examination for the French Graduate Engineering Schools. During my internships, I had the opportunity to discover various fields of mobile robotics such as agriculture robotics with Tecnoma, vineyard robotics with Exxact Robotics, and underwater robotics with Forssea Robotics. I am then very familiar with the industrial application of robotics.

		During my studies, I learned the basics of robotics such as robot mechanical conception, hardware design, simulation, and low-level control. I was also able to build a solid foundation in information technology through courses on applied mathematics, image and signal processing, machine learning, embedded operating systems. Additionally, I was able to reinforce my knowledge in robotics by learning control theory, probabilistic and set methods for state estimation, navigation, and path planning.

		Thanks for your time and consideration. I genuinely believe that my experience and education would make me a valuable asset to the mechatronics research group. I believe my skills and motivation make me a great potential asset. I can be reached by phone and email if you need any further information.

		\lettersection{Introduction}
		
			I would consider pursuing my studies by doing a Ph.D. at Tampere University mechatronics research group to be a privilege. I have already met Pr. Kari Koskinen and Dr. Jussi Aaltonen, and I had the opportunity to visit the laboratory and present some of my works. From what I was able to discover during my visit, my profile and my skills are perfectly suited to the expectations of the mechatronics research group, with projects that were presented to me such as PVTO, RoboMiners or the AutoFeeder project, which this thesis will be part of.

			Pr. Luc Jaulin, teacher and researcher at ENSTA Bretagne, also wishes to follow my thesis work in order to bring his expertise in the control of mobile robots, especially in the application of set methods. 
			
		\lettersection{Problem Statement}

			The control of systems in a guaranteed and resilient way still has many shortcomings. This thesis will allow to remove scientific locks in the fields of control and command of systems in a safe and guaranteed way.

			\begin{itemize}[noitemsep,topsep=0pt,parsep=0pt,partopsep=0pt]
				\item How the guarantee provided by interval analysis can help us to realize a safe command-control loop?
				\item How to ensure collision-free control of a system in a dynamic environment?
				\item How can we ensure the security and resilient control of a system that operates in degraded mode?
			\end{itemize}

		\lettersection{Research Methods}
		
		\lettersection{Preliminary Schedule}

			After a state of the art review in the field of guaranteed and resilient control of systems, as well as in the classical control of surface vessels, I will alternate between phases of modeling and simulation and phases of real tests to validate the theoretical tools implemented. The results of this theoretical work on the control of systems and their application to the control of autonomous surface vessels through trial results will be subject to publications.
			
			As requested by the Ph.D. offer, at least four publications will be scheduled. The guaranteed control algorithms as well as those in degraded situations could be published in an international journal such as Oceanic Engineering or IEEE Transaction on Robotics. Two international conferences in robotics are also targeted \textsc{ICRA} and \textsc{IROS} for example.

			Finally, the last year will be mainly devoted to the compilation of the results, the formalization of the implemented tools and the writing of my thesis as well as the preparation of my thesis defense.

			\resizebox*{\textwidth}{!}{
				\begin{ganttchart}[
						y unit title=0.5cm, y unit chart=0.7cm, vgrid,hgrid,
						title height=1, title label font=\bfseries\footnotesize,
						bar/.style={fill=awesome-skyblue}, bar height=0.7, inline]{1}{48}

					\gantttitle{Preliminary Schedule}{48}\\
					\gantttitle[]{2022}{6} \gantttitle[]{2023}{12} \gantttitle[]{2024}{12} \gantttitle[]{2025}{12} \gantttitle[]{2026}{6} \\
					\gantttitle{Q3}{3} \gantttitle{Q4}{3}
					\gantttitle{Q1}{3} \gantttitle{Q2}{3} \gantttitle{Q3}{3}  \gantttitle{Q4}{3}
					\gantttitle{Q1}{3} \gantttitle{Q2}{3} \gantttitle{Q3}{3} \gantttitle{Q4}{3}
					\gantttitle{Q1}{3} \gantttitle{Q2}{3} \gantttitle{Q3}{3} \gantttitle{Q4}{3}
					\gantttitle{Q1}{3} \gantttitle{Q2}{3} \\

					\ganttbar[inline=false]{State of the Art}{1}{6}
					\ganttbar[inline=false]{}{16}{18}
					\ganttbar[inline=false]{}{28}{30} \\

					\ganttbar[inline=false]{Modelling and Simulation}{4}{12}
					\ganttbar[inline=false]{}{16}{21}
					\ganttbar[inline=false]{}{28}{33} \\

					\ganttbar[inline=false]{Laboratory and Field Experiments}{10}{15}
					\ganttbar[inline=false]{}{22}{27}
					\ganttbar[inline=false]{}{34}{42} \\

					\ganttbar[inline=false]{Article and Thesis}{10}{12}
					\ganttbar[inline=false]{}{22}{24}
					\ganttbar[inline=false]{}{34}{36}
					\ganttbar[inline=false]{}{40}{48}
				\end{ganttchart}
			}

		\lettersection{Conclusion}

		\makeletterclosing

	\end{cvletter}

	% 

	% 	\lettersection{A~propos de moi}
	% 	Je suis Quentin \textsc{Brateau}, élève ingénieur à l'\textsc{ENSTA} Bretagne, Ecole Nationale Supérieure de Techniques Avancées de Bretagne, basée à Brest. Je serai diplomé en septembre ingénieur en Robotique Mobile. Je suis actuellement en contrat de professionnalisation avec l'entreprise \textsc{Forssea Robotics} pour cette dernière année de formation. Cette expérience professionnelle m'a permis de découvrir la robotique industrielle dans une start-up, et de maitriser \textsc{ROS} et \textsc{Gazebo} afin de réaliser un simulateur pour leur \textsc{ROV}s.

	% 	\lettersection{Pourquoi Moi ?}
	% 	Etant futur ingénieur roboticien, spécialisé dans la robotique mobile, ce sujet de thèse s'inscrit parfaitement dans la continuité de ma formation. De plus, j'ai eu l'opportunité de faire parti de l'équipe de l'eurathlon 2019-2020 de l'\textsc{ENSTA} Bretagne, concours dans lequel des drones aériens et des drones terrestres collaborent pour cartographier un milieu endommagé. Cette année, j'ai aussi pu diriger un projet industriel soutenu par \textsc{Forssea Robotics} dans lequel un drone devait attérir de manière autonome sur un robot terrestre en mouvement. Je suis donc très familier avec le fonctionnement et le contrôle classique des drones.

	% 	\lettersection{}
	% 	Finissant mon contrat de professionnalisation fin septembre, je serais disponible pour commencer cette thèse début octobre 2021.

		

	% 

	

	

\end{document}
