%!TEX TS-program = xelatex
%!TEX encoding = UTF-8 Unicode

\documentclass[11pt, a4paper]{awesome-cv}

\usepackage[french]{babel}

\geometry{left=1.4cm, top=.8cm, right=1.4cm, bottom=1.8cm, footskip=.5cm}
\fontdir[fonts/]
\colorlet{awesome}{awesome-skyblue}
\setbool{acvSectionColorHighlight}{true}

% If you would like to change the social information separator from a pipe (|) to something else
\renewcommand{\acvHeaderSocialSep}{\quad\textbar\quad}

% Available options: circle|rectangle,edge/noedge,left/right
% \photo[circle,noedge,left]{./examples/profile}
\name{Quentin}{Brateau}
\position{Ingénieur Robotique{\enskip\cdotp\enskip}Systèmes embarqués}
\address{17 rue Adolphe Laberte 51100 Reims}

\mobile{(+33) (0)6 04 18 24 28)}
\email{quentin.brateau@ensta-bretagne.org}
\homepage{teusner.github.io/portfolio}
\github{Teusner}
\linkedin{quentinbrateau}
% \gitlab{gitlab-id}
% \stackoverflow{SO-id}{SO-name}
% \twitter{@twit}
% \skype{skype-id}
% \reddit{reddit-id}
% \medium{medium-id}
% \googlescholar{googlescholar-id}{name-to-display}
%% \firstname and \lastname will be used
% \googlescholar{googlescholar-id}{}
% \extrainfo{extra informations}

% \quote{``Be the change that you want to see in the world."}

% The company being applied to
\recipient
  {Thèse AID - 2022850}
  {Cartographie et localisation simultanées avec données aberrantes à l'aide du calcul par intervalles}
% The date on the letter, default is the date of compilation
\letterdate{\today}
% The title of the letter
\lettertitle{Candidature de thèse en robotique sous-marine}
% How the letter is opened
\letteropening{Madame, Monsieur,}
% How the letter is closed
\letterclosing{Veuillez agréer, Madame, Monsieur, l'assurance de ma considération distinguée,}
% Any enclosures with the letter

\begin{document}

	% Give optional argument to change alignment(C: center, L: left, R: right)
	\makecvheader[C]

	\makecvfooter{\today}{Quentin Brateau~~~·~~~Lettre de Motivation}{}

	\makelettertitle

	\begin{cvletter}

		Je suis Quentin \textsc{Brateau}, ingénieur en robotique diplômé de l'\textsc{ENSTA} Bretagne en 2021, Ecole Nationale Supérieure de Techniques Avancées de Bretagne, basée à Brest. J'ai obtenu la même année le diplôme du Master Recherche de l'Université d'Angers dans la spécialité Système Dynamiques et Signaux afin de poursuivre mes études en thèse.
		
		Après ma dernière année d'école d'ingénieur effectuée en alternance chez \textit{Forssea Robotics}, j'ai été embauché à l'\textsc{ENSTA} Bretagne en tant qu'ingénieur de recherche dans des sujets très liés à la robotique marine. Ces deux expériences fortes m'ont fait découvrir les spécificités ainsi que comprendre les enjeux stratégiques de la robotique marine et sous-marine. En effet, j'ai pu travailler sur de la simulation de propagation acoustique sous-marine, ce qui a mené à des travaux liés à de la localisation de source acoustiques par des méthodes ensemblistes, ainsi que sur des sujets de navigation garantie et résiliente pour des navires autonomes.

		J'aimerai donc aujourd'hui poursuivre mes études en faisant une thèse en robotique sous-marine. Le sujet proposé conjointement par l'\textsc{AID} et le laboratoire de robotique de l'\textsc{ENSTA} Bretagne m'intéresse particulièrement, car il allie robotique sous-marine et méthodes ensemblistes en un sujet qui me semble attrayant et très prometteur. Cette thèse s'inscrira dans la continuité des travaux déjà menés à l'\textsc{ENSTA} Bretagne, dont j'ai pu prendre connaissance lors de mon passage en tant qu'élève puis en tant qu'ingénieur de recherche au laboratoire de robotique et qui ont su capter mon intérêt.

	\end{cvletter}

	\makeletterclosing

\end{document}
