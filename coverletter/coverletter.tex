%!TEX TS-program = xelatex
%!TEX encoding = UTF-8 Unicode

\documentclass[11pt, a4paper]{awesome-cv}

\usepackage{pgfgantt}
\usepackage{graphicx}
\usepackage{xcolor}
\usepackage{enumitem}

\geometry{left=1.4cm, top=.8cm, right=1.4cm, bottom=1.8cm, footskip=.5cm}
\fontdir[fonts/]
\colorlet{awesome}{awesome-skyblue}
\setbool{acvSectionColorHighlight}{true}

% If you would like to change the social information separator from a pipe (|) to something else
\renewcommand{\acvHeaderSocialSep}{\quad\textbar\quad}

% Available options: circle|rectangle,edge/noedge,left/right
% \photo[circle,noedge,left]{./examples/profile}
\name{Quentin}{Brateau}
\position{Robotics Engineer{\enskip\cdotp\enskip}Embedded Systems}
\address{17 rue Adolphe Laberte 51100 Reims}

\mobile{(+33) (0)6 04 18 24 28)}
\email{quentin.brateau@ensta-bretagne.org}
\homepage{teusner.github.io/portfolio}
\github{Teusner}
\linkedin{quentinbrateau}
% \gitlab{gitlab-id}
% \stackoverflow{SO-id}{SO-name}
% \twitter{@twit}
% \skype{skype-id}
% \reddit{reddit-id}
% \medium{medium-id}
% \googlescholar{googlescholar-id}{name-to-display}
%% \firstname and \lastname will be used
% \googlescholar{googlescholar-id}{}
% \extrainfo{extra informations}

% \quote{``Be the change that you want to see in the world."}

% The company being applied to
\recipient
  {\textsc{Tampere University}}
  {Kalevantie 4, 33100 Tampere}
% The date on the letter, default is the date of compilation
\letterdate{\today}
% The title of the letter
\lettertitle{Ph.D. Letter of Intent : Safe and resilient control of autonomous surface vessels in dynamic environments}
% How the letter is opened
\letteropening{Dear Madam/Sir,}
% How the letter is closed
\letterclosing{I hope you will consider my application favourably and I look forward to hearing from you. \\[\baselineskip] Yours sincerly,}
% Any enclosures with the letter
% \letterenclosure[Pièces Jointes]{
% 	\begin{itemize}
% 		\item Curriculum Vitae
% 		\item Lettre de recommandation du professeur Luc \textsc{Jaulin}
% 		\item Lettre de recommandation d'Alaa \textsc{El Jawad} tuteur de mon projet de fin d'études
% 		\item Relevés de notes obtenus à l'\textsc{ENSTA} Bretagne
% 	\end{itemize}~}

\begin{document}

	% Give optional argument to change alignment(C: center, L: left, R: right)
	\makecvheader[C]

	\makecvfooter{\today}{Quentin Brateau~~~·~~~Letter of Intent}{}

	\makelettertitle

	\begin{cvletter}

		I am Quentin Brateau, a French engineer in robotics. I was graduated in 2021 ENSTA Bretagne, in Brest, a prominent French graduate, post-graduate school, Research Institute, and doctoral college, which trains engineers in various fields of engineering such as Robotics, which is my specialty. I also was graduated the same year with a Master of Science in dynamical systems and signals from the University of Angers in France.

		I would like to apply for one of the 4 fully-funded Ph.D. you are offering in your Doctoral Program in Engineering Sciences. I am particularly interested in working with the mechatronics research team. This offer seems particularly attractive to me because it has a strong industrial focus and allows for concrete applications of the research conducted.

		My academic training has a strong emphasis on Science because I have completed a two-year intensive post-secondary school preparation for the competitive entrance examination for the French Graduate Engineering Schools. During my internships, I had the opportunity to discover various fields of mobile robotics such as agriculture robotics with Tecnoma, vineyard robotics with Exxact Robotics, and underwater robotics with Forssea Robotics. I am then very familiar with the industrial application of robotics.

		During my studies, I learned the basics of robotics such as robot mechanical conception, hardware design, simulation, and low-level control. I was also able to build a solid foundation in information technology through courses on applied mathematics, image and signal processing, machine learning, embedded operating systems. Additionally, I was able to reinforce my knowledge in robotics by learning control theory, probabilistic and set methods for state estimation, navigation, and path planning.

		Thanks for your time and consideration. I genuinely believe that my experience and education would make me a valuable asset to the mechatronics research group. I believe my skills and motivation make me a great potential asset. I can be reached by phone and email if you need any further information.

		\lettersection{Introduction}
		
			I would consider to be a privilege pursuing my studies by doing a Ph.D. at Tampere University mechatronics research group. I have already met Pr. Kari Koskinen and Dr. Jussi Aaltonen, and I had the opportunity to visit the laboratory and present some of my works. From what I was able to discover during my visit, my profile and my skills are perfectly suited to the expectations of the mechatronics research group, with projects that were presented to me such as PVTO, RoboMiners or the AutoFeeder project, which this thesis will be part of.

			Pr. Luc Jaulin~\footnote{\url{https://www.ensta-bretagne.fr/jaulin}}, teacher and researcher at ENSTA Bretagne, also wishes to follow my thesis work in order to bring his expertise in the control of mobile robots, especially in the application of set methods. 
			
		\lettersection{Problem Statement}

			The uncertainties related to the maritime environment are many and come from different factors such as capricious weather, unknown currents the risk of collision with offshore infrastructures or other surface vessels, the possibility of a breakdown with the impossibility to repair on site due to the absence of operators, and many other reasons that make it difficult to explore this marine environment.

			The control of systems in a guaranteed and resilient way still has many shortcomings. This thesis will allow to remove scientific locks in the fields of control and command of systems in a safe and guaranteed way.

			\begin{itemize}[noitemsep,topsep=0pt,parsep=0pt,partopsep=0pt]
				\item How the guarantee provided by interval analysis can help us to realize a safe command-control loop?
				\item How to ensure collision-free control of a system in a dynamic environment?
				\item How can we ensure the security and resilient control of a system that operates in degraded mode?
			\end{itemize}

		\lettersection{Research Methods}

			Autonomous surface vessels control is a field that seems very promising and still needs a lot of work to guarantee the security of the systems. In particular, this area has shortcomings in the control of ships in a dynamic world. This goes from high-level trajectory planning to last-second avoidance of moving obstacles. To ensure that vessels do not collide with each other or with moving objects, guaranteed tools must be used. In particular, we have to get rid of probabilistic methods which provide the most probable state of the system and use set methods~\footnote{Applied Interval Analysis: With Examples in Parameter and State Estimation, Robust Control and Robotics, Jaulin, L. and Kieffer, M. and Didrit, O. and Walter, E., 2012} which provide us the set of possible states for the system. 

			The state equations for marine and underwater vehicles are well known~\footnote{Handbook of Marine Craft Hydrodynamics and Motion Control, Fossen, T.I., 2021}. After having developed a twin model of the system as close as possible to the real surface vessel, we will be able on the one hand to build controllers and observers of the system based on this model, but also to simulate the behavior of the system and thus to test the theoretical tools before their implementation on the real autonomous surface vessel. We could also get from this simulator a higher-level dynamic behavior prediction to know in advance the robustness and performance of the system.
			
			In the context of large autonomous surface vessels control, this is even more important since it is necessary to be sure before launching a mission autonomously that the algorithms are working properly. Software such as Ignition Gazebo or Vortex Studio are very powerful and offer the possibility to simulate the behavior of robots in dynamic and complex environments.
			
			Both of them are also compatible with the ROS2 simulation framework. This middleware provides the foundations of the software architecture that could be used in this thesis. Indeed, in addition to the ease of development, ROS2 also allows to interchange the simulator with the drivers allowing to communicate with the real robot, and thus it facilitates the passage from prototyping to testing by launching the same code in these two phases.
			
			Then, in a context of guarantee and safety of operation of autonomous surface vessels, it seems necessary to carry out resilient control. Indeed, the boat must be able on the one hand to continue its mission despite the loss of a sensor or an actuator. This can be temporary, as with the loss of the GNSS signal, or permanent with the loss of communication with a sensor or the shutdown of an actuator for example. On the other hand, the boat must be able to judge if it no longer has enough information from sensors or not enough control on actuators to continue its mission without any collisions in a guaranteed way.
		 
			Finally, after having set up all the software architecture and having been able to simulate the behavior of the various algorithms implemented, it seems imperative to perform trials on real boats to validate the various tools and methods that have been set up during this Ph.D.
		
		\lettersection{Preliminary Schedule}

			After a state of the art review in the field of guaranteed and resilient control of systems, as well as in the classical control of surface vessels, I will alternate between phases of modeling and simulation and phases of real tests to validate the theoretical tools implemented. The results of this theoretical work on the control of systems and their application to the control of autonomous surface vessels through trial results will be subject to publications.
			
			As requested by the Ph.D. offer, at least four publications will be scheduled. The guaranteed control algorithms as well as those in degraded situations could be published in an international journal such as Oceanic Engineering or IEEE Transaction on Robotics. Two international conferences in robotics are also targeted, for instance \textsc{ICRA} and \textsc{IROS}.

			Finally, the last year will be mainly devoted to the compilation of the results, the formalization of the implemented tools and the writing of my thesis as well as the preparation of my thesis defense.

			\resizebox*{\textwidth}{!}{
				\begin{ganttchart}[
						y unit title=0.5cm, y unit chart=0.7cm, vgrid,hgrid,
						title height=1, title label font=\bfseries\footnotesize,
						bar/.style={fill=awesome-skyblue}, bar height=0.7, inline]{1}{48}

					\gantttitle{Preliminary Schedule}{48}\\
					\gantttitle[]{2022}{6} \gantttitle[]{2023}{12} \gantttitle[]{2024}{12} \gantttitle[]{2025}{12} \gantttitle[]{2026}{6} \\
					\gantttitle{Q3}{3} \gantttitle{Q4}{3}
					\gantttitle{Q1}{3} \gantttitle{Q2}{3} \gantttitle{Q3}{3}  \gantttitle{Q4}{3}
					\gantttitle{Q1}{3} \gantttitle{Q2}{3} \gantttitle{Q3}{3} \gantttitle{Q4}{3}
					\gantttitle{Q1}{3} \gantttitle{Q2}{3} \gantttitle{Q3}{3} \gantttitle{Q4}{3}
					\gantttitle{Q1}{3} \gantttitle{Q2}{3} \\

					\ganttbar[inline=false]{State of the Art}{1}{6}
					\ganttbar[inline=false]{}{16}{18}
					\ganttbar[inline=false]{}{28}{30} \\

					\ganttbar[inline=false]{Modelling and Simulation}{4}{12}
					\ganttbar[inline=false]{}{16}{21}
					\ganttbar[inline=false]{}{28}{33} \\

					\ganttbar[inline=false]{Laboratory and Field Experiments}{10}{15}
					\ganttbar[inline=false]{}{22}{27}
					\ganttbar[inline=false]{}{34}{42} \\

					\ganttbar[inline=false]{Article and Thesis}{10}{12}
					\ganttbar[inline=false]{}{22}{24}
					\ganttbar[inline=false]{}{34}{36}
					\ganttbar[inline=false]{}{40}{48}
				\end{ganttchart}
			}

		\lettersection{List of Publications}

			\begin{itemize}
				\item Acoustic source localisation in underwater environment using interval analysis, Quentin Brateau, Benoit Zerr, Luc Jaulin, Proceedings of ICUA 2022 (Submitted)
			\end{itemize}

		\makeletterclosing

		% References avec footnote + References de contacts

	\end{cvletter}
\end{document}
